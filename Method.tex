\section{Method}

%\bigskip 
%ger stort mellanrum \medskip ger lite mindre mellanrum \smallskip ännu mindre
%\paragraph{Här skriver man vad sin paragraph ska heta, det används mer då man vill dela upp sin text i olika delstycken typ om man ska förklara elektriska komponenter så blir varje delstycke en komponent 


\paragraph{}
The administration of the project was composed of a conjoint Google Drive storage sharing, Trello application for task planning and management together with the application Slack for all the communication.

It was decided that the group would have a minimum of two meetings a week, at the beginning and end of each week, to review and discuss the work done so far along with planning goals and tasks for the upcoming week. At the end of each week, the group would try to have a meeting with the supervisor at Cybercom, where he would review the progress of the project, answer questions from the group members and give advise or help with the program software. Other project related questions and problems were directed to the Chalmers' supervisor, who has been helping with the way to plan and organise the project for us to be able to complete it in time.

\paragraph{}
Before getting access to anything from the company all group members signed a standard non-disclosure agreement because the software that was going to be handed out to potentially be worked with contained confidential information. The group members also got access to the company's Github account with relevant information about the project. We received a log file with annexed documentation, to be parsed. 

\paragraph{}
The group decided on a specific case scenario to solve and with the help of Scrum the case was broken down into smaller, more feasible tasks. 

Our original method to test our V2X scenario was supposed to use the software supplied by Cybercom, but after some weeks we had to rethink our plans. The new strategy was to create a database with which to parse the information we acquired from the pseudo-log file with the purpose to map the locations of the two entities registered in the file and study the gathered data at each point. 

\paragraph{}
To achieve this a database schema was costructed capable of holding the data and a parser to read the log into the databas was written in java according to the following methodology.

By mapping each value on the line with its ID as key in a hashmap the data for each column in the database is parsed as it is found. This handles all the top-level IDs but needs special adjustment for the "location" and "blackboardProperties" entries.

By splitting the "location" entry into "longitude" and "latitude" in the line's hashmap it is handled with the rest.

Storing "blackboardProperties" requires more adjustments, since there can be any number of entries per line. To manage this a separate database table holds the property entries. The number of the parent line is entered as the identifier for the parent line and the data in the line's hashmap is inserted into the database when reaching "blackboardProperties"; since it is the last  type, not stored with the rest and depends on the parent line existing in the database (as seen in the database schema). In "blackboardProperties" the data for "type" and value (ID varies) are inserted into a new hashmap with their IDs as keys. Each "blackboardProperties" entry is inserted into the database when fully read. It is entered with the line number of the parent as "parent", value of "type" as "type" and the value of the remaining key (key ignored) as "value".

Thus, when reaching the end of the line all data has been parsed and inserted into the database.
