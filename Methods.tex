\section{Methods}
\subsection{Det här är också en underrubrik}
\subsubsection{Det här är underrubrikens rubrik}

\bigskip 
%ger stort mellanrum \medskip ger lite mindre mellanrum \smallskip ännu mindre
%\paragraph{Här skriver man vad sin paragraph ska heta, det används mer då man vill dela upp sin text i olika delstycken typ om man ska förklara elektriska komponenter så blir varje delstycke en komponent }

The project was started by creating a common database to make working together easier for all group members. It was decided that the group would have a minimum of two meetings a week, at the beginning and end of each week to review and discuss the work done so far and to put up goals/ tasks for the upcoming week. At the end of each week, the group would have a meeting with the supervisor at Cybercom, where he would check how the project is going, answer questions from the group members and give advise or help with the program software. Slack was used as a professional source of communication between all group members and so far it has made internal daily group contact simple and efficient. To organise the work and distribute the workload in a clear way between group members the Trello application was used with an influence of the Scrum method to solve the tasks. The group decided on  a specific case scenario to solve and with the help of Scrum the scenario was broken down into smaller tasks. In Trello the case scenario was documented together with the user cases in a backlog, and from there these user cases were divided  into smaller more concrete problems to solve. For each week up until now the group decided on a few tasks which would be distributed between all group members with, and at the beginning of the next week the  potential results would be discussed between group members and problems would be brought up to solve together. 
