\documentclass{article}
\usepackage[utf8]{inputenc}
\usepackage[english]{babel}
\usepackage{graphicx}
\graphicspath{ {images/} }
\usepackage{float}
\usepackage[T1]{fontenc}
\usepackage{amsfonts}
\usepackage{amsmath}
\usepackage{verbatimbox}
\usepackage{titlepic}
\usepackage{cite}
\usepackage{url}
\usepackage{geometry}
\usepackage{hyperref}
\usepackage{titling}
\usepackage{blindtext}
\usepackage{imakeidx}
\usepackage{listings}
\lstset{
   breaklines=true,
   basicstyle=\ttfamily}

 
\hypersetup{
    colorlinks=true,
    linkcolor=black,
    filecolor=black,      
    urlcolor=black,
}
\urlstyle{same}

\begin{document}
\font\myfont=cmr12 at 32pt

\begin{titlepage}

\title{\vspace{7ex} { {\myfont V2X - feature development}} \vspace{50ex} }
%\includegraphics[scale=0.4]{images/titlepange.png}


\author{ Tedd Ahlberg, Erik Benjaminsson, Filip Kaiser \\ Fressia Merino,  Shahad Naji, David Olofsson}
\date{January 2019}
\end{titlepage}

%titel page no numbering  
\clearpage\maketitle
\thispagestyle{empty}

\cleardoublepage
\setcounter{page}{2}
\pagenumbering{Roman}			% Arabic numbering starting from 1 (one)
\setlength{\parskip}{0pt plus 1pt}


\section{Abstract}
To achieve a safer and more efficient mode of transport communication vehicles need to understand their surroundings. This report strives to describe how this can be done with ITS-G5 at a pedestrian crossing. The conclusions herein are likely to be applicable to other similar applications, such as traffic lights and railway crossings.

\newpage

% start index page 
\tableofcontents


\cleardoublepage
\setcounter{page}{3}
\pagenumbering{arabic}			% Arabic numbering starting from 1 (one)
\setlength{\parskip}{0pt plus 1pt}



\newpage
\section{Introduction}
V2X, short for vehicle-to-everything, is a technology meant to make traffic safer by letting the car communicate with its surroundings, covering up some of the blind spots in the drivers spacial perception and understanding. There are different types of communications that can be implemented in real life situations such as V2I (vehicle to infrastructure), V2G (vehicle to grid), V2D (vehicle to device), V2V (vehicle to vehicle), V2P (vehicle to pedestrian), V2N (vehicle to network) and V2TT (vehicle to tram and train). In this project we will focus on a special case of V2I where a car is moving towards a crossing in a simulation environment. This case makes us detect and understand some risks and problems of real world situations (\textit{see figure 1}).

\begin{figure}[H]
    \centering
    \includegraphics[scale=0.45]{images/V2I.png}
    \caption{Example on V2I}
\end{figure}

\paragraph{}
V2X communication means that vehicles exchange information with each
other and the nearby infrastructure. This new concept is associated with Intelligent Transportation Systems (ITS), a technology with the aim to reduce traffic jam, the environmental impact of transportation and most importantly, to reduce lethal traffic accidents. To enable this technology, wireless communication is needed. Therefore a system that communicates with cars, infrastructure or other vehicles to lower the percentage of fatal accidents is very sought-after. Vehicle manufacturers have been developing this technology for a long time but right now they are facing problems with latency. The existing system depends on a cloud service which has too high latency and this presents a difficult problem in situations where a microsecond could make a big difference in saving the lives of road travellers.

\newpage
\subsection{Purpose}
The purpose of the project is to make a pilot-study for a future development of "V2X" for Cybercom, therefore we decided to make a simple program that can parse information to a database from a log file. To apply in a real world situation we chose to have a back story, see Figure 1, where a pedestrian has pressed the button on a crosswalk traffic light and is about to cross the street. Our task is to structure a future implementation of the V2I program which warns both the pedestrian and the car that either one of them is approaching/trying to cross a crosswalk.

\subsection{Limitations}
The limitations for this project is that only the basic implementation of V2I will be tested. V2V is already a working concept and won't be altered. There will be no safety checks either. The ambition of the project is to test the implementation of V2I in forms of reading a log file and to compare two positions acquired from the file. It will be limited to parse information from the log file supplied by Cybercom. In the beginning of the project the software OTTO and Oscar - OTTO is a simulation environment and Oscar is a program used to communicate through ITS-G5 - should have been supplied by Cybercom, but due to lack of time and secrecy we chose not to use them.  

\section{Background}
This project will be a pilot-study for future work because the time that we have will not be enough to finish the implementation. The V2V system present today has very high latency and will not be as precise when travelling through a cloud. Therefore many car manufacturers want to create an even better system for a safer way of travel.
\paragraph{}
Cybercom, which is the company that we are doing this for, is a consultant company that has other partners who develop the V2V feature of V2X. Cybercom wants to help these companies by testing if there are other parts in a traffic situation where this type of communication can come in hand.
They have therefore sent out this project for Chalmers University of Technology to investigate if there are other ways this system can be implemented for a safer travel for everyone who travels on land.
\section{Theory}

In today's automotive companies huge resources are put in optimising and developing efficient processes that are in favour of the environmental, safety on the roads and antonyms cars. In this cluster of new technology is something called "V2V" \textit{(Vehicle-to-vehicle communication)}. V2V is a wireless transmission system between cars on the road. The reason and idea behind this technology is to get cars around each other to communicate with each other to reduce the risk of both collision and more energy effective driving. The information that is sent between the cars are their speed, position and direction.

\begin{figure}[H]
    \centering
    \includegraphics{images/V2VGlobal.png}
    \caption{Cloud network}
\end{figure}

This information is sent over an AD-HOC network. AD-HOC is a local area network (LAN) that's used in close area networking. This is used instead of a global network “cloud network”. Since a global network would mean that every car needs to be connected over satellite via some service or through the phone via Bluetooth. This is not a reliably way. A cloud network also adds an delay in the signal because the travel distance in increased. So a LAN network is much safer in theory. \bigskip

AD-HOC is used for many reasons but mostly because a global network can have delay between the server and the cars and is hard to use in areas with bad connection to the grid. By using AD-HOC the delay in the signals between the vehicles are reduced and in traffic a shorter delay can save lives.
\bigskip

The goal of V2V is to reduce the total of cars crashes and especially in the future when cars are becoming completely autonomous and driving by themselves. This is where the V2X project enters the picture, \textit{"Because if cars are to communicate with each other, why not let the cars communicate with everything on the road like bikes, traffic lights and pedestrian crossings. Is that possible?"} It had reduced the risk of collision even more.
\bigskip

\begin{figure}[H]
    \centering
    \includegraphics[scale=0.4]{images/projektsketch.jpg}
    \caption{LAN network}
\end{figure}

The technology used is called "Cooperative Intelligent Transport Systems" (C-ITS).  “ITS-G5” is a broadcast technology based on an evolution of the wireless standard 802.11p. It is the only validated and available technology on the market and capable of delivering secure AD-HOC direct vehicle-to-vehicle and/or vehicle-to-infrastructure communication. ITS-G5 is running in the designated 5.9 GHz frequency band that is foreseen for road safety. We emphasise that all technologies that run in this frequency band should not cause interference with each other and be interoperable.C-ITS should be able to demonstrate its capability to co-exist with electronic road charging, the enforcement of drive and rest times and weights and dimensions on the adjacent 5.8 GHz frequency band. [1]

\section{Method}

%\bigskip 
%ger stort mellanrum \medskip ger lite mindre mellanrum \smallskip ännu mindre
%\paragraph{Här skriver man vad sin paragraph ska heta, det används mer då man vill dela upp sin text i olika delstycken typ om man ska förklara elektriska komponenter så blir varje delstycke en komponent 


\paragraph{}
The administration of the project was composed of a conjoint Google Drive storage sharing, Trello application for task planning and management together with the application Slack for all the communication.

It was decided that the group would have a minimum of two meetings a week, at the beginning and end of each week, to review and discuss the work done so far along with planning goals and tasks for the upcoming week. At the end of each week, the group would try to have a meeting with the supervisor at Cybercom, where he would review the progress of the project, answer questions from the group members and give advise or help with the program software. Other project related questions and problems were directed to the Chalmers' supervisor, who has been helping with the way to plan and organise the project for us to be able to complete it in time.

\paragraph{}
Before getting access to anything from the company all group members signed a standard non-disclosure agreement because the software that was going to be handed out to potentially be worked with contained confidential information. The group members also got access to the company's Github account with relevant information about the project. We received a log file with annexed documentation, to be parsed. 

\paragraph{}
The group decided on a specific case scenario to solve and with the help of Scrum the case was broken down into smaller, more feasible tasks. 

Our original method to test our V2X scenario was supposed to use the software supplied by Cybercom, but after some weeks we had to rethink our plans. The new strategy was to create a database with which to parse the information we acquired from the pseudo-log file with the purpose to map the locations of the two entities registered in the file and study the gathered data at each point. 
\section{Results}

We were able to create a database scheme, included in the appendix, to hold the data in an easily filtered form. Additionally a parser, also included in the appendix, that reads one logfile into a database following the scheme has been written in Java. 
The database schema and parser have been confirmed functional with the 4.7 MB log-file we received from Cybercom and an off-site database. The filtering and searching functions simplify the work with the data gathered in the log-file for a future development of the project. The intended future development would be a graphical program that visualises the data which would further improve the understanding of what this data signifies. 

Below we present a test string from the log-file which the script inserts into the database:

\begin{lstlisting}
{ "itemId": "103636965585520737", "stationId": "1", "appId": "123", "typeId": "OSCAR_BLACKBOARD_TYPE_VEHICLE_STATUS", "location": { "latitude": 52.5228688988356, "longitude": 13.410204655793073 }, "locationConfidence": 1, "validityDuration": 1.5, "validityArea": "NaN", "blackboardProperties": [{"type": "OSCAR_BLACKBOARD_PROPERTY_ACCELERATION", "acceleration": 0 }, {"type": "OSCAR_BLACKBOARD_PROPERTY_SPEED", "speed": 10 }, { "type": "OSCAR_BLACKBOARD_PROPERTY_YAW_RATE", "yawRate": 0 } ] }}
\end{lstlisting}

\begin{figure}[H]
    \centering
    \includegraphics[scale=0.3]{images/db-screenshot.jpg}
    \caption{Screen-shot from database}
\end{figure}
\section{Discussion}
For future work it is recommended to have documents, such as documentation about the project and NDA:s, prepared and ready for the first meeting with the assigned group and to have the software ready to be handed out. That would have helped the speed and progress of the project. Though, as mentioned before, with the circumstances that we had we think that our achievement is a good start for farther development of the V2X research. 
\paragraph{}
We chose not to create a test program based on out assumptions about the software without taking a look at it and without knowing its structure and what information we could receive. Instead we decided to focus on writing theoretically about the techniques of V2X and creating a database that simplifies the assessing of information from the log-file that we got from Cybercom and potentially easing the way for future developers to analyse this kind of information.

\paragraph{}
The time distribution was a problem mostly because it was very difficult for us to get the files and the program that we initially planned to work with. When we finally got the files that we needed there was little time left for us to analyse it and work on our scenario, which resulted in adding more limitations to our work, by scoping down to focusing on parsing the log file into a database. This caused our work to be very theoretical instead of being practical like we had planned from the start. We were anticipating from the start to have access to OTTO to be able to simulate a test to develop a basic implementation of our case scenario. Therefore, as we mentioned in the result, we were only able to create a database and construct a basic implementation which stores information from the log file in a database. 

\paragraph{}
The cooperation between all the group members in the project worked well and we did everything that we could to create something useful with the available material. There were some areas of improvement in some aspects such as time planning and work distribution because it was our first time working with this type of project so it was a little hard to know how to organise the work and set up a working schedule that fits all group members. 




\section{Conclusion}

Coming in a few weeks (aiming for week 49/50)

\newpage
\paragraph{References}
\begin{verbatim}
[1] - Documentation on \textit{ITS-G5 is ready to roll}, Available:
      https://itsg5-ready-to-roll.eu/} [Accessed: 10-11-18] 

[2] - From Wikipedia, the free encyclopedia, \textit{Local area network}, Available: 
      https://en.wikipedia.org/wiki/Local_area_network [Accessed: 28-11-18] 

[3] - From Wikipedia, the free encyclopedia, \textit{Wide Area Network}, Available: 
      https://sv.wikipedia.org/wiki/Wide_Area_Network [Accessed: 28-11-18] 

[4] - Cybercom Group a IT consulting company, Available:
      https://www.cybercom.com/sv/ [Accessed: 01-11-18]
\end{verbatim}

\newpage
\paragraph{Appendix}

\paragraph{A1. Database schema}
\begin{verbatim}
CREATE TABLE blackboard (
       key SERIAL PRIMARY KEY,
       itemid VARCHAR,
       stationid INTEGER,
       appid INTEGER,
       typeid VARCHAR,
       latitude FLOAT,
       longitude FLOAT,
       locationconfidence FLOAT,
       validityduration FLOAT,
       validityarea INTEGER --Currently unused--                                              
);

CREATE TABLE properties (
       parent INTEGER,
       type VARCHAR,
       value VARCHAR,

       PRIMARY KEY (parent, type),
       FOREIGN KEY (parent) REFERENCES blackboard
);
\end{verbatim}

\newpage
\paragraph{A2. Log-file parser}
\begin{verbatim}

import java.io.FileInputStream;
import java.io.InputStreamReader;
import java.io.BufferedReader;
import java.lang.StringBuilder;
import java.util.HashMap;
import java.sql.*; // JDBC stuff.
import java.util.Properties;


public class OsCarLogParser
{
    static String USERNAME = "";
    static String PASSWORD = "";

    //Reads input from command line and opens the local logfile to be parsed.
    //Hands over to parse() while keeping track of used resources and closes when parse() is done.
    public static void main(String[] args) throws Exception
    {
	FileInputStream fs = null;
	InputStreamReader fr = null;
	BufferedReader file = null;
	//If filename is entered, use it. Else, print error
	if ( args.length <= 2 )
	    {
		System.out.println("Enter the relative path to the log file, the username and the password.");
	    }
	else
	    {
		USERNAME = args[1];
		PASSWORD = args[2];
		
		try
		    {
			//open the file in an easily used format
			fs = new FileInputStream(args[0]);
		        fr = new InputStreamReader(fs);
			file = new BufferedReader(fr);
			
			parse(file);
		    }
		catch(Exception e)
		    {
			System.out.println("Error occured opening file. Please enter an existing, readable file.");
			e.printStackTrace();
		    }
		finally
		    {
			// releases system resources associated with this stream
			if(file!=null)
			    {
				file.close();
			    }
			if(fr!=null)
			    {
				fr.close();
			    }
			if(fs!=null)
			    {
				fs.close();
			    }	
		    }
		return;
	    }
    }

    //An error macro for parse errors
    private static void error()
    {
	System.out.println("Unknown log format!");
	System.exit(1);
    }

    //The method that does the rest, could be split into database and parse.
    private static void parse(BufferedReader logfile)
    {
	//Do the database thing
	Connection conn = null;
	try
	    {
		//Note that this will fail if the driver jar isn't in the classpath
		Class.forName("org.postgresql.Driver");
		String url = "jdbc:postgresql://localhost/";
		Properties props = new Properties();
		props.setProperty("user",USERNAME);
		props.setProperty("password",PASSWORD);
		conn = DriverManager.getConnection(url, props);
	    }
	catch( Exception e )
	    {
		System.out.println("Error occured connecting to database. Have you set the right connection settings in the source-code? Do you have network access to the database?");
		e.printStackTrace();
		System.exit(1);
	    }

	if (conn == null)
	    {
		error();
	    }

	System.out.println("Database connected");
	
	//Read the file line by line, parsing each line
	try
	    {
		HashMap<String,String> blackboard = new HashMap<String,String>();
		HashMap<String,String> property = new HashMap<String,String>();
		HashMap<String,String> map = blackboard;
		StringBuilder buffer = new StringBuilder();
		String key = new String();
		Boolean inProperties = false;
		        
		int linenr = 0;
		for (String line = logfile.readLine(); line != null; line = logfile.readLine()) {
		    System.out.println(++linenr);
		    //First check that the line starts correctly with "{"
		    if (line.charAt(0) != '{')
			{
			    error();
			}
		    //Loop through the line setting the data into a buffer.
		    //Every data string into its own place in the hash-map.
		    //Properties are placed in a separate hash-map, so that the varying names of the columns can be handled.
		    blackboard = new HashMap<String,String>();
		    property = new HashMap<String,String>();
		    map = blackboard;
		    buffer = new StringBuilder();
		    key = new String();
		    inProperties = false;
		    for ( int i = 1; line.length() > i; i++ )
			{
			    char buff = line.charAt(i);
			    switch (buff)
				{
				case ' ':
				    //Blank spaces have no meaning in the log, thus they are ignored
				    break;
				case '"':
				    //End or start of quote are extraneous, since other symbols declare the start and end of strings
				    break;
				case ':':
				    //The column name for the data is in the buffer. Save it to key.
				    key = buffer.toString();
				    //Clear the buffer to make space for the data.
				    buffer = new StringBuilder();
				    break;
				case ',':
				    //Denominates the end of a key-value pair. Save the data into map.
				    map.put(key, buffer.toString());
				    //And clear the buffer
				    buffer = new StringBuilder();
				    break;
				case '{':
				    //This is a underlying data pair. These are either split and saved with the rest or put into a separate map
				    if (key.equals("location") )
					{
					    //These are longitude and latitude or 
					    //Pretend everything is normal, the key will be overwritten and the sub data entered with the rest
					}
				    else if ( inProperties )
					{
					    //These are properties. They are to be put into the properties map, which must be cleared.
					    property = new HashMap<String,String>();
					    map = property;
					    //As above the key will be overwritten and data entered normally
					}
				    else
					{
					    //This should never happen!
					    System.out.println("Misplaced {! character " + i);
					    error();
					}
				    break;
				case '}':
				    //If this is in properties, commit the map to the database
				    //Else, ignore it
				    if ( inProperties )
					{
					    //Bind the last given value into the map
					    map.put(key, buffer.toString());
					    //And clear the buffer
					    buffer = new StringBuilder();
					    
					    //Initiate query
					    PreparedStatement query = conn.prepareStatement("INSERT INTO properties VALUES(?, ?, ?)");
					    query.setInt(1, linenr);
					    for( HashMap.Entry<String, String> tmp : map.entrySet() )
						{
						    if (tmp.getKey().equals("type"))
							{
							    //enter as type in database
				       			    query.setString(2, tmp.getValue());
							}
						    else
							{
							    //enter as value in database
							    query.setString(3, tmp.getValue());
							}
						}
					    //execute query
					    query.executeUpdate();
					}
				    break;
				case '[':
				    //Start of the underlying properties, just make sure.
				    if (key.equals("blackboardProperties") )
					{
					    //blackboardProperties MUST be the last entry. Therefore, save the object and get the int key
					    
					    //Initiate query
					    PreparedStatement query = conn.prepareStatement("INSERT INTO blackboard VALUES( ?, ?, ?, ?, ?, ?, ?, ?, ?, ?)");
					    //Set the key
					    query.setInt(1, linenr);
					    //Insert all the data
					    for( HashMap.Entry<String, String> tmp : map.entrySet() )
						{
						    switch (tmp.getKey())
							{
							case "itemId":
							    query.setString(2, tmp.getValue());
							    break;
							case "stationId":
							    query.setInt(3, Integer.parseInt(tmp.getValue()));
							    break;
							case "appId":
							    query.setInt(4, Integer.parseInt(tmp.getValue()));
							    break;
							case "typeId":
							    query.setString(5, tmp.getValue());
							    break;
							case "latitude":
							    query.setFloat(6, Float.parseFloat(tmp.getValue()));
							    break;
							case "longitude":
							    query.setFloat(7, Float.parseFloat(tmp.getValue()));
							    break;
							case "locationConfidence":
							    query.setFloat(8, Float.parseFloat(tmp.getValue()));
							    break;
							case "validityDuration":
							    query.setFloat(9, Float.parseFloat(tmp.getValue()));
							    break;
							case "validityArea":
							    //query.setInt(10, Integer.parseInt(tmp.getValue()));
							    query.setInt(10, 0);
							    break;
							}
						}
					    //execute query
					    query.executeUpdate();//query stuff
					    //query.
					    //Set the inProperties flag and roll along. The first data pair will handle setting the map correctly.
					    inProperties = true;
					}
				    else
					{
					    //This should never happen!
					    System.out.println("Misplaced [!");
					    error();
					}
				    break;
				case ']':
				    //Mark that we are leaving properties
				    if (inProperties)
					{
					    inProperties = false;
					}
				    break;
				default:
				    buffer.append(line.charAt(i));
				    break;
					    
				}
			}
		}
	    }
	catch(Exception e)
	    {
		e.printStackTrace();
		System.out.println("Error occured when reading file.");
	    }
    }
}

\end{verbatim}


\end{document}


%lite kommentarer
%i conclusion börja kanske inte vad man inte kunde klara av, 
%skriv typ, syftet var såhär och sen komma in på vad vi inte klarade av, även vad vi har klarat av. osv

%visa mer i resultat i resultat, skriv vad man hade kunnat göra med datan
%beskriv koden, referera till den i appendix, hur ser 

%diskussion, vilka var limitations? skriv mer så att någon annan förstår

%conclusion vilka papers? -kontrakt, dokumentation osv

%resultat-hur går man vidare?

Jens:

hur går man vidare?

hur tycker vi att det var att börja med projektet? hur har det känts?