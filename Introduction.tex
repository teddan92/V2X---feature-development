\newpage
\section{Introduction}
V2X is short for vehicle-to-everything, it is meant to make traffic safer by letting the car communicate with surroundings and make the driver more aware. There is different types of communication such as V2I (vehicle to infrastructure), V2G (vehicle to grid), V2D (vehicle to device), V2V (vehicle to vehicle), V2P (vehicle to pedestrian) and V2N (vehicle to network). In this project we will focus on a special case of V2I where a car moving towards a crossing is simulated. This is to understand risks and problems. ITS-G5 will be used as a communication protocol, it is a standard wifi connection. Because standard wifi is already tested, widely implementable and cheap it can be used for safety in vehicles. To simulate access point (wifi) a software osCar is used and implemented in a simulation environment called otto. OsCar V2X software framework is a complete high level API with possibility to integrate with car head-unit to display warnings. Otto is a test environment made by Cybercom to test different scenarios with osCar. 