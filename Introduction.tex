\section{Introduction}
\subsection{Det här är en underrubrik? :)}
Backgroud - Vehicle-to-everything (V2X) communication is the passing of information from
a vehicle to any entity that may affect the vehicle, and vice versa. It is a vehicular
communication system that incorporates other more specific types of communication as 
V2I (Vehicle to Infrastructure), V2N (Vehicle-to-network), V2V (Vehicle-to-vehicle), V2P (Vehicle-to-Pedestrian), V2D (Vehicle-to-device) and V2G (Vehicle-to-grid) 
%́The main motivations for V2X are road safety, traffic efficiency, and energy savings.

Introduktion till ITS-G5, varför använda ?, var används den? vilken MHz(mer theory)
Tanken med V2X är att man skall få snabbare responstid än med cloud 

V2X utvecklas för att minska olyckor i trafiken, det finns olika typer av V2X\bigskip



V2X is short for vehicle-to-everything, it is meant to make traffic safer by letting the car communicate with surroundings and make the driver more aware. There is different types of communication such as V2I (vehicle to infrastructure), V2G (vehicle to grid), V2D (vehicle to device), V2V (vehicle to vehicle), V2P (vehicle to pedestrian) and V2N (vehicle to network). In this project we will focus on a special case of V2I where a moving car moving towards a crossing is simulated, to understand risks and problems. ITS-G5 will be used as a communication protocol, it is a standard wifi connection. Because standard wifi is already tested, widely implementable and cheap it can be used for safety in vehicles. To simulate access point (wifi) a software osCar is used and implemented in a simulation environment called otto. OsCar V2X software framework is a complete high level API 